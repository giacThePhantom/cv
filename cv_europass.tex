\documentclass[english,a4paper]{europasscv}
\usepackage{multicol}

\ecvname{Giacomo Fantoni}
\ecvaddress{Via Grazia Deledda 42, Ghedi (BS), 25016, Italy}
\ecvmobile{(+39) 328 118 2628}
\ecvemail{giacomo.fantoni@studenti.unitn.it}
\ecvgithubpage{https://github.com/giacThePhantom}
\ecvlinkedinpage{https://www.linkedin.com/in/giacomo-fantoni-68b800235}
\ecvdateofbirth{11 October 1999}
\ecvnationality{Italian}
\ecvgender{Male}

\begin{document}
\begin{europasscv}
   \ecvpersonalinfo
   \ecvitem{}{
      An energetic and highly motivated professional with a strong commitment to achieving objectives in a timely manner while maintaining a focus on excellence.
      I have a keen interest in Computational Biology, specifically within the realm of Computational Neuroscience.
      I offer a proactive and disciplined approach to problem-solving, coupled with exceptional communication and leadership abilities gained through various group projects during both Master's and Bachelor's studies.
      Additionally, I am eager to broaden my skill set by acquiring experimental laboratory work skills, as I believe a comprehensive understanding of both computational and experimental techniques will enhance my ability to analyze neuronal data and make significant contributions in the field.
   }

   \ecvsection{Education}

   \ecvtitle{September 2021 -- Present}{Master's Degree in Quantitative and computational biology}
   \ecvitem{}{University of Trento, Trento (Italy)}
   \ecvitem{Master thesis}{\textit{``Synchronization of neuronal activity in the antennal lobe of the honeybee during sleep''}, Supervisor Prof. A. Haase.}
   \ecvitem{Relevant Coursework}{
      \begin{ecvitemize}
         \item Bioinformatics.
         \item Machine Learning.
         \item Mathematical Modeling.
         \item Computational Biophysics.
         \item Function and plasticity of the Central Nervous System.
      \end{ecvitemize}
   }

   \ecvtitle{September 2018 -- September 2021}{Bachelor's Degree in Computer Science}
   \ecvitem{}{University of Trento, Trento (Italy)}
   \ecvitem{Bachelor thesis}{\textit{``Analysis of trascriptomic RNA-seq data from polisomial and total fraction from an epithelial cancer cellular line''}. Supervisor Prof. A. Inga. Final grade 108/110}
   \ecvitem{Relevant Coursework}{
      \begin{ecvitemize}
         \item Molecular biology.
         \item Genetics.
      \end{ecvitemize}
   }

   \ecvsection{Experience}
   \ecvtitle{March 2021 -- August 2021}{Intern -- Analysis of transcriptomic RNA-seq data}
   \ecvitem{}{University of Trento, Department of CIBIO, Trento (Italy)}
   \ecvitem{}{
      \begin{ecvitemize}
        \item Implemented a pipeline to process RNA-seq cancer data on a remote server.
        \item Analysed processed data using a combination of R and python to gain biological insights.
        \item The work is accessible at \ecvhighlight{\href{https://github.com/giacThePhantom/Tesi}{https://github.com/giacThePhantom/Tesi}}.
      \end{ecvitemize}
   }
   \ecvtitle{March 2021 -- August 2021}{Intern -- Computational modelling and simulation of the antennal lobe of the honeybee}
   \ecvitem{}{University of Trento, Department of CiMeC, Trento (Italy)}
   \ecvitem{}{
      \begin{ecvitemize}
         \item Implemented a gpu-accelerated computational model of the antennal lobe of the honeybee in python using the \ecvhighlight{\href{https://genn-team.github.io}{genn}} library.
         \item Analysed model results to gain biological insights from it.
      \end{ecvitemize}
   }

   \ecvsection{Personal projects}
   \ecvtitle{}{Personal coursework collection}
   \ecvitem{}{
      Wrote in \LaTeX\ all the material regarding the content of my university courses, making it accessible at \ecvhighlight{\href{https://github.com/giacThePhantom}{https://github.com/giacThePhantom}}.
      This has provided easily accessible quality material for my colleagues and it has encouraged some people to join this project, creating a small community of maintainers.
   }
   \ecvtitle{}{Implementation of the Morris-Lecar neuron model}
   \ecvitem{}{
      A group project dealing with neuron dynamics implemented in Matlab.
      The code models the dynamics of a neuron using an hybrid stochastic model in which the membrane potential evolves according to a deterministic differential equation and the opening and closing of ion channels are modelled as a stochastic process.
      Source code is available at: \ecvhighlight{\href{https://github.com/giacthephantom/mathematical-modeling-and-simulation-project}{https://github.com/giacthephantom/mathematical-modeling-and-simulation-project}}.
   }
   \ecvtitle{}{Reimplementation of the tool ``Current based decomposition -- CURBD''}
   \ecvitem{}{
      Re-implemented \ecvhighlight{\href{https://github.com/rajanlab/CURBD}{CURBD}}, a Python tool that trains a spiking recurrent neural network capable of reproducing experimental neural data.
      This reimplementation offers a more flexible,  modular, and explainable codebases.
      It enables the tool to leverage inferred functional interactions to from the trained models to uncover directional currents between multiple brain regions.
      The code can be accessed at \ecvhighlight{\href{https://github.com/giacThePhantom/CURBD}{https://github.com/giacThePhantom/CURBD}}.
   }
   \ecvsection{Personal skills}

      \ecvmothertongue{Italian}
      \ecvlanguageheader
      \ecvlanguage{English}{C1}{C1}{C1}{C1}{C1}
      \ecvlanguagefooter

      \ecvdigitalcompetence
         {\ecvProficient}
         {\ecvProficient}
         {\ecvIndependent}
         {\ecvProficient}
         {\ecvProficient}

      \ecvitem{Programming skills}{
         \begin{ecvitemize}
            \item Python (numpy, pandas, scikit-learn, pytorch).
            \item R.
            \item C/C++.
            \item Java.
            \item Bash.
            \item Matlab.
            \item Javascript.
            \item Linux systems.
            \item Git.
            \item \LaTeX.
         \end{ecvitemize}
      }
\end{europasscv}
\end{document}
