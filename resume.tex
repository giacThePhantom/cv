\documentclass{resume} % Use the custom resume.cls style

\usepackage[left=0.4 in,top=0.4in,right=0.4 in,bottom=0.4in]{geometry} % Document margins
\newcommand{\tab}[1]{\hspace{.2667\textwidth}\rlap{#1}}
\newcommand{\itab}[1]{\hspace{0em}\rlap{#1}}
\name{Giacomo Fantoni} % Your name
% You can merge both of these into a single line, if you do not have a website.
\address{+39(328) 118-2628 \\ Via Delle Orfane 2, 38122, Trento, Italy \\ \href{mailto:giacomo.fantoni@studenti.unitn.it}{giacomo.fantoni@studenti.unitn.it}}

\begin{document}

\begin{rSection}{}
      An energetic and highly motivated professional with a strong commitment to achieving objectives in a timely manner while maintaining a focus on excellence.
      I have a keen interest in Computational Biology, specifically within the realm of Computational Neuroscience.
      I offer a proactive and disciplined approach to problem-solving, coupled with exceptional communication and leadership abilities gained through various group projects during both Master's and Bachelor's studies.
      Additionally, I am eager to broaden my skill set by acquiring experimental laboratory work skills, as I believe a comprehensive understanding of both computational and experimental techniques will enhance my ability to analyze neuronal data and make significant contributions in the field.
\end{rSection}
%----------------------------------------------------------------------------------------
%	OBJECTIVE
%----------------------------------------------------------------------------------------

% \begin{rSection}{OBJECTIVE}

% {Software Engineer with 2+ years of experience in XXX, seeking full-time XXX roles.}


% \end{rSection}
%----------------------------------------------------------------------------------------
%	EDUCATION SECTION
%----------------------------------------------------------------------------------------

\begin{rSection}{Education}

{\bf Universit\`a degli studi di Trento} \hfill {Trento, Italy}\\
Master of Quantitative and Computational Biology \hfill {Sept. 2021 - Present}
\begin{itemize}
   \item Relevant coursework: Bioinformatics, Data mining, Mathematical Modeling, Computational Biophysics, Function and plasticity of the Central Nervous System.
\end{itemize}

{\bf Universit\`a degli studi di Trento} \hfill {Trento, Italy}\\
Bachelor of Computer Science\hfill {Sept. 2018 - Sept. 2021}\\
BA Thesis: "Analysis of trascriptomic RNA-seq data from polisomial and total fraction from an epithelial cancer cellular line". Supervisor: Prof. A. Inga. Final grade 108/110.


\end{rSection}


\begin{rSection}{EXPERIENCE}

\textbf{Universit\`a degli studi di Trento, Department of CIBIO} \hfill March 2021 - August 2021\\
Analysis of trascriptomic RNA-seq data \hfill \href{https://github.com/giacThePhantom/Tesi}{github.com/giacThePhantom/Tesi} \hfill \textit{Trento, Italy}\\
 \begin{itemize}
    \itemsep -3pt {}
     \item Implemented a pipeline to process RNA-seq data on a remote server.
     \item Analysed processed data using both R and python to gain biological insights regarding their involvement in cancer.
 \end{itemize}

\textbf{Universit\`a degli studi di Trento, Department of CIMEC} \hfill March 2023 - Present\\
Computational modelling and simulation of the antennal lobe of the Honey Bee to uncover neuronal correlates of sleep \hfill \href{https://github.com/giacThePhantom/genn-network-model}{github.com/giacThePhantom/genn-network-model} \hfill \textit{Rovereto, Italy}
 \begin{itemize}
    \itemsep -3pt {}
     \item Implemented a gpu-accelerated computational model of the antennal lobe in python.
     \item Implemented a machine learning pipeline to analyse the data produced by the model.
 \end{itemize}

\end{rSection}

%----------------------------------------------------------------------------------------
%	WORK EXPERIENCE SECTION
%----------------------------------------------------------------------------------------

\begin{rSection}{PROJECTS}
\vspace{-1.25em}
\item \textbf{Personal coursework collection}{
      Wrote in \LaTeX\ all the material regarding the content of my university courses, making it accessible at {\href{https://github.com/giacThePhantom}{github.com/giacThePhantom}}.
      This has provided easily accessible quality material for my colleagues and it has encouraged some people to join this project, creating a small community of maintainers.
   }
\item \textbf{Implementation of the Morris-Lecar neuron model}
   {
      A group project dealing with neuron dynamics implemented in Matlab.
      The code models the dynamics of a neuron using an hybrid stochastic model in which the membrane potential evolves according to a deterministic differential equation and the opening and closing of ion channels are modelled as a stochastic process.
      Source code is available at: {\href{https://github.com/giacthephantom/mathematical-modeling-and-simulation-project}{github.com/giacthephantom/mathematical-modeling-and-simulation-project}}.
   }
\item \textbf{Reimplementation of the tool ``Current based decomposition -- CURBD''}
   {
      Re-implemented {\href{https://github.com/rajanlab/CURBD}{CURBD}}, a Python tool that trains a spiking recurrent neural network capable of reproducing experimental neural data.
      This reimplementation offers a more flexible,  modular, and explainable codebases.
      It enables the tool to leverage inferred functional interactions to from the trained models to uncover directional currents between multiple brain regions.
      The code can be accessed at {\href{https://github.com/giacThePhantom/CURBD}{github.com/giacThePhantom/CURBD}}.
   }
\end{rSection}

%----------------------------------------------------------------------------------------

%----------------------------------------------------------------------------------------
% TECHINICAL STRENGTHS
%----------------------------------------------------------------------------------------
\begin{rSection}{SKILLS}

\begin{tabular}{ @{} >{\bfseries}l @{\hspace{6ex}} l }
Programming: & Python (numpy, pandas, scikit-learn, pytorch), R, C/C++, Java, bash, Matlab, \\
 &Javascript, Linux systems, Git, \LaTeX\\
Languages: & Fluent in Italian and English
\end{tabular}\\
\end{rSection}


\end{document}
